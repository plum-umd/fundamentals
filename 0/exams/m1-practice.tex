\documentclass[12pt]{article}
\usepackage{fullpage}
\usepackage{times}
\usepackage{xcolor,censor}
\usepackage{fontspec}
\usepackage{fancyvrb}

\setmainfont{Times New Roman}

\censorruleheight=0ex
\StopCensoring

\begin{document}

\begin{center}
  {\Large {\bf CMSC 198Q, Midterm 1 (PRACTICE)}\\
    \censor{{\bf SOLUTION}} \ \\
    \ \\
    Summer 2019 \ \\
  }
\end{center}

\vspace{2em}

\noindent
NAME:\verb|________________________________________________________|



\vspace{2em}

\begin{center}
\begin{tabular}{| c | c |}
  \hline
  Question & Points \\ \hline \hline
  1 & 10 \\ \hline
  2 & 10 \\ \hline
  3 & 15 \\ \hline
  4 & 10 \\ \hline
  5 & 20 \\ \hline
  Total: & 65\\
  \hline
\end{tabular}
\end{center}

\noindent
This test is open-book, open-notes, but you may not use any computing
device other than your brain and may not communicate with anyone.
You have 60 minutes to complete the test.

\vskip 1em

\noindent
The phrase ``design a program'' or ``design a function'' means follow
the steps of the design recipe.  Unless specifically asked for, you do
not need to provide intermediate products like templates or stubs,
though they may be useful to help you construct correct solutions.

\vskip 1em

\noindent
You may use any of the data definitions given to you within this exam
and do not need to repeat their definitions.

\vskip 1em

\noindent
Unless specifically instructed otherwise, you may use any built-in BSL
functions or data types.

\vskip 1em

\noindent
When writing tests, you may use a shorthand for writing check-expects
by drawing an arrow between two expressions to mean you expect the
first to evaluate to same result as the second.  For example, you may
write \verb|(add1 3)| $\rightarrow$ \verb|4| instead of
\verb|(check-expect (add1 3) 4)|.

\newpage

\noindent
{\bf Problem 1 (10 points).}
%
For the following program, write out each step of computation. At each
step, underline the expression being simplified. Label each step as
being ``arithmetic'' (meaning any built-in operation),
``conditional'', ``plug'' (for plugging in an argument for a function
parameter), or ``constant'' for replacing a constant with its value.


\begin{verbatim}
(define Q 1)
(define (w z) (< (string-length z) 20))
(w (cond [(= 1 Q) "fred"]
         [else 9]))
\end{verbatim}


\begin{SaveVerbatim}{VerbEnv}  

SOLUTION:

(w (cond [(= 1 Q) "fred"] [else 9])) -->[const]
               ^
(w (cond [(= 1 1) "fred"] [else 9])) -->[arith]
          ^^^^^^^
(w (cond [#true "fred"] [else 9]))   -->[cond]
   ^^^^^^^^^^^^^^^^^^^^^^^^^^^^^^
(w "fred")                           -->[plug]
^^^^^^^^^^
(< (string-length "fred") 20))       -->[arith]
   ^^^^^^^^^^^^^^^^^^^^^^
(< 5 20)                             -->[arith]
#true
\end{SaveVerbatim}

%\censor{\vspace{-2.5in}}
\censor{%
\noindent\BUseVerbatim{VerbEnv}}

\newpage

\noindent
{\bf Problem 2 (10 points).}
%
For the following structure definition, list the names of every
function it creates.  For each function, classify it as being either a
constructor, accessor, or predicate.

\begin{verbatim}
(define-struct dr (who strange))
\end{verbatim}

\begin{SaveVerbatim}{VerbEnv}  
SOLUTION:

- make-dr    : Constructor
- dr-who     : Accessor
- dr-strange : Accessor
- dr?        : Predicate
\end{SaveVerbatim}

%\censor{\vspace{-2.5in}}
\censor{%
\noindent\BUseVerbatim{VerbEnv}}

\newpage

\noindent
{\bf Problem 3 (15 points).}
%
You've been hired by the CMNS development office and been given their
existing software for spamming potential donors.  It contains the following
data definition:
\begin{verbatim}
;; A Person is a (make-name Title String String)
;; Interp: a person's title and first and last names.
(define-struct name (title first last))

;; A Title is one of:
;; - "r"      Interp: Mr.
;; - "s"              Ms.
;; - "d"              Dr.
\end{verbatim}

\noindent
Finish the design of the following function for creating letter
openings:
\begin{verbatim}
;; dear : Person -> String
;; Create a letter opening for the given person.
(check-expect (dear (make-name "d" "Minnie" "Maisy"))
              "Dear Dr. Maisy:")
(check-expect (dear (make-name "r" "Fred" "Rogers"))
              "Dear Mr. Rogers:")
(check-expect (dear (make-name "s" "Miriam" "Maisel"))
              "Dear Ms. Maisel:")
\end{verbatim}


\begin{SaveVerbatim}{VerbEnv}

SOLUTION:

(define (dear p)
  (string-append "Dear " 
                 (title-abbrv (person-title p)) 
                 " " 
                 (person-last p) 
                 ":"))

;; title-string : Title -> String
;; Render title as a string abbreviation
(check-expect (title-abbrv "d") "Dr.")
(check-expect (title-abbrv "r") "Mr.")
(check-expect (title-abbrv "s") "Ms.")
(define (title-abbrv t)
  (cond [(string=? t "d") "Dr."]
        [(string=? t "r") "Mr."]
        [(string=? t "s") "Ms."]))
\end{SaveVerbatim}

\censor{%
\noindent
\BUseVerbatim{VerbEnv}}

\newpage
\noindent
{\bf Problem 4 (10 points).}
%
Do you ever mispell words like ``peice'' or ``acheive'' (which are
correctly spelled ``piece'' and ``achieve'')?  Turns out a lot of
people make \emph{transposition} errors like this when typing.  Based
on this insight, you decide to make a next generation messaging app
that helps users by giving them a \emph{transpose} operation.  The
idea is that as a user types, they can place their cursor between two
letters that need to be transposed and invoke {\tt transpose}.  To
implement this feature, you define the following function for
transposing letters at a given position in a string that has at
least 2 letters in it:
\begin{verbatim}
;; transpose : String Index -> String
;; Transpose characters at left and right of index i.
;; Assumes string has length >= 2 and 1<=i, i+1<=length.
(check-expect (transpose "ab" 1) "ba")
(check-expect (transpose "peice" 2) "piece")
(check-expect (transpose "student's" 8) "students'")
(define (transpose s i) ...)
\end{verbatim}

\noindent
Give a correct definition for \verb|transpose|.  (You only need to
provide code, not the design steps.)

\begin{SaveVerbatim}{VerbEnv}

SOLUTION:
  
(define (transpose s i)
  (string-append (substring s 0 (sub1 i))
                 (substring s i (add1 i))
                 (substring s (sub1 i) i)
                 (substring s (add1 i))))
\end{SaveVerbatim}

\censor{%
\noindent
\BUseVerbatim{VerbEnv}}

\newpage
\noindent
{\bf Problem 5 (20 points).}  Growing tired of Bug, you
decide to build a new video game.  Part of the game consists of flying
billiard balls that move in straight lines at varying velocities until
hitting other balls or bouncing off walls and other obstacles.  After
making some sketches, you decide on the following data representation
for billiard balls:

\begin{verbatim}
;; A BB is a:
;; (make-bb Color 
;;          (make-posn Integer Integer) 
;;          (make-vel Integer Integer))

;; A Color is one of:
;; - "red"
;; - "yellow"
;; - "green"

(define-struct bb (color center vel))
(define-struct vel (deltax deltay))
\end{verbatim}
The interpretation of a \verb|BB| is that the color is the color of
the ball, the \verb|posn| is the location of the center of the ball,
and the \verb|vel| structure describes the ball's velocity as a change
along the $x$-axis (\verb|deltax|) and $y$-axis (\verb|deltay|) in one
clock tick.

\vskip 1em

\noindent
Design a function called \verb|tock : BB -> BB| that calculates where
a given billiard ball will be, based on its velocity, after one tick
of the clock and assuming it does not encounter any obstacle.

\begin{SaveVerbatim}{VerbEnv}

SOLUTION:
  
;; tock : BB -> BB
;; Move ball one tick based on velocity
(check-expect (tock (make-bb "red" (make-posn 2 5) (make-vel 1 -2)))
              (make-bb "red" (make-posn 3 3) (make-vel 1 -2)))
(define (tock bb)
  (make-bb (bb-color bb)
           (make-posn (+ (posn-x (bb-center bb))
                         (vel-deltax (bb-vel bb)))
                      (+ (posn-y (bb-center bb))
                         (vel-deltay (bb-vel bb))))
           (bb-vel bb)))
\end{SaveVerbatim}

\censor{%
\noindent
\BUseVerbatim{VerbEnv}}


\end{document}

