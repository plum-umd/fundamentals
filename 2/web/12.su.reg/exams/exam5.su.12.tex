\documentclass[11pt]{article}
\usepackage{amsmath}
%% ------------------------------------------------------------------
%% SOLUTIONS:
\def\thel{\noindent\rule{2.5cm}{.5pt}}
\long\def\begsol#1 #2\endsol{#1}
%\def\begsol#1{\thel {\bf Solution} \thel}\def\endsol{\relax}
%%uncomment the above line to get the solutions printed

\newcommand\code[1]{\texttt{#1}}
\newcommand\bcode[1]{\texttt{\textbf{#1}}}

%% PROBLEMS:
\def\pts#1{\marginpar{\footnotesize \raggedright  \fbox{#1 {\sc Points}}}}
\newcounter{Pctr}
\newenvironment{problem}{\stepcounter{Pctr}%
\begin{description}
\item[\noindent{\bf Problem} \arabic{Pctr}] 
\end{description}}{\relax}
%% ------------------------------------------------------------------

%% SUBPARTS:
\newcounter{parts}
\renewcommand{\theparts}{\Alph}
%% ------------------------------------------------------------------

\begin{document}

\renewcommand{\theenumi}{\Alph{enumi}}
\setcounter{Pctr}{0}

%% ------------------------------------------------------------------

\vfill
\centerline{\Large CS 2510 Exam 5 -- Summer 2012}

~\\[2cm]

\begin{center}
\begin{tabular}{l@{\qquad}l}
Name:                        & \rule{200pt}{.1pt} \\[.5cm]
Student Id (last 4 digits):  & \rule{200pt}{.1pt} \\[.5cm]
\end{tabular}
\end{center}

\noindent\begin{minipage}{7.5cm} $\bullet$ Write down the answers in the
space provided. 

$\bullet$ You may use all syntax of Java that we have studied in
class.

$\bullet$ For tests you only need to provide the expression that
computes the actual value, connecting it with an arrow to the expected
value. For example \code{s.method() -> true} is sufficient.

$\bullet$ Remember that the phrase ``design a class'' or ``design a
method'' means more than just providing a definition. It means to
design them according to the \textbf{design recipe}.  You are
\textit{not} required to provide a method template unless the problem
specifically asks for one.  However, be prepared to struggle if you
choose to skip the template step.

\bigskip

\textit{Good luck!}
\end{minipage}\hfil\begin{minipage}[t]{4.5cm}
\begin{tabular}{|c|l@{\qquad\qquad}|r|}
\hline
\textbf{Score} &  & 45 \\ \hline
\end{tabular}
\end{minipage}

\vfill\thispagestyle{empty}
\newpage

%% -----------------------------------------------------------------------------
%% 
\pts{45}
\begin{problem}
Develop an implementation of {\tt Comparator<Posn>} that orders
Cartesian-coordinates by their distance from the origin in such a way
that the further away from the origin, the smaller the position is
considered.  So for example, there is no position larger than (0,0).

You may rely on the following definition of {\tt Posn}:

\begin{verbatim}
// Represents (x,y) in Cartesian-coordinate system
class Posn {
  Integer x;
  Integer y;
  Posn(Integer x, Integer y) {
    this.x = x;
    this.y = y;
  }
}
\end{verbatim}
\end{problem}

\newpage
\begin{problem}
Design a method:
\begin{center}
{\tt <T> Boolean isSorted(ArrayList<T> ls, Comparator<T> c)}
\end{center}
that determines if the given array list is in ascending sorted
order according to the given comparator.
\end{problem}
\newpage
\begin{problem}
Here is a data definition for pairs:

\begin{verbatim}
// Represents a pair of A and B.
class Pair<A,B> {
  A left;
  B right;
  Pair(A left, B right) {
    this.left = left;
    this.right = right;
  }
}
\end{verbatim}
Design a method:
\begin{verbatim}
<A,B> Comparator<Pair<A,B>> lexi(Comparator<A> ca,
                                 Comparator<B> cb)
\end{verbatim}
that consumes a comparator for {\tt A}s and a comparator for {\tt B}s
and produces a comparator for pairs of {\tt A}s and {\tt B}s that is
the \emph{lexicographic order}.  Mathematically speaking, the
lexicographic order of a pair is defined as follows:
$$
(a,b) \leq (a',b')\text{ if and only if }a < a'\text{ or }(a = a'\text{ and }b \leq b')
$$
\end{problem}
\ 
\newpage
\ 
\newpage
\ 
\end{document}

