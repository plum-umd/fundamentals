\documentclass[11pt]{article}
\usepackage{amsmath}
%% ------------------------------------------------------------------
%% SOLUTIONS:
\def\thel{\noindent\rule{2.5cm}{.5pt}}
\long\def\begsol#1 #2\endsol{#1}
%\def\begsol#1{\thel {\bf Solution} \thel}\def\endsol{\relax}
%%uncomment the above line to get the solutions printed

\newcommand\code[1]{\texttt{#1}}
\newcommand\bcode[1]{\texttt{\textbf{#1}}}

%% PROBLEMS:
\def\pts#1{\marginpar{\footnotesize \raggedright  \fbox{#1 {\sc Points}}}}
\newcounter{Pctr}
\newenvironment{problem}{\stepcounter{Pctr}%
\begin{description}
\item[\noindent{\bf Problem} \arabic{Pctr}] 
\end{description}}{\relax}
%% ------------------------------------------------------------------

%% SUBPARTS:
\newcounter{parts}
\renewcommand{\theparts}{\Alph}
%% ------------------------------------------------------------------

\begin{document}

\renewcommand{\theenumi}{\Alph{enumi}}
\setcounter{Pctr}{0}

%% ------------------------------------------------------------------

\vfill
\centerline{\Large CS 2510 Exam 6 -- Summer 2012}

~\\[2cm]

\begin{center}
\begin{tabular}{l@{\qquad}l}
Name:                        & \rule{200pt}{.1pt} \\[.5cm]
Student Id (last 4 digits):  & \rule{200pt}{.1pt} \\[.5cm]
\end{tabular}
\end{center}

\noindent\begin{minipage}{7.5cm} $\bullet$ Write down the answers in the
space provided. 

$\bullet$ You may use all syntax of Java that we have studied in
class.

$\bullet$ For tests you only need to provide the expression that
computes the actual value, connecting it with an arrow to the expected
value. For example \code{s.method() -> true} is sufficient.

$\bullet$ Remember that the phrase ``design a class'' or ``design a
method'' means more than just providing a definition. It means to
design them according to the \textbf{design recipe}.  You are
\textit{not} required to provide a method template unless the problem
specifically asks for one.  However, be prepared to struggle if you
choose to skip the template step.

\bigskip

\textit{Good luck!}
\end{minipage}\hfil\begin{minipage}[t]{4.5cm}
\begin{tabular}{|c|l@{\qquad\qquad}|r|}
\hline
\textbf{Score} &  & 45 \\ \hline
\end{tabular}
\end{minipage}

\vfill\thispagestyle{empty}
\newpage

%% -----------------------------------------------------------------------------
%% 
\begin{center}
A useful utility
\end{center}
You may assume, \textbf{for the purposes of examples only}, the existence of a
class {\tt Iter<X>} that implements {\tt Iterable<X>}.  It has a
somewhat magical constructor in that it can take \emph{any} number of
{\tt X}s as arguments.  Its {\tt iterator} method produces an iterator
that iterates through those arguments.

For example:
\begin{verbatim}
new Iter<Integer>().iterator().hasNext()  --> false
new Iter<Integer>(1).iterator().hasNext() --> true
new Iter<Integer>(1).iterator().next()    --> 1
Iterator<Integer> i =
  new Iter<Integer>(1,2,3).iterator();
i.next() --> 1
i.next() --> 2
i.next() --> 3
\end{verbatim}
\newpage
\pts{45}
\begin{problem}
Design a method
\begin{verbatim}
    Double avg(Iterable<Double> nums)
\end{verbatim}
that computes the average of a series of numbers.  It should raise an
exception if the series is empty.
\end{problem}
\newpage
\begin{problem}
Design a class
\begin{verbatim}
   class SumSquares implements Iterable<Integer>
\end{verbatim}
which is given an {\tt Iterable<Integer>} when constructed and whose
{\tt iterator} method produces an iterator that produces the sum of
the squares the given iterator produces.

For example, if given a {\tt new Nat()}, the iterable you wrote in lab
that produces the natural numbers {\tt 0, 1, 2, 3, 4}, \dots, then
iterator should produce {\tt 0, 1, 5, 14, 30}, etc.

\end{problem}
\newpage
\begin{problem}
Design a class
\begin{verbatim}
   class Zip<X,Y> implements Iterable<Pair<X,Y>>
\end{verbatim}
which is given an {\tt Iterable<X>} and an {\tt Iterable<Y>} when
constructed and whose {\tt iterator} method produces an iterator
that produces pairs of values: one from the {\tt X} series, one from
the {\tt Y} series.  The iterator should have no more elements
whenever either of the given iterators have no more elements.

You may rely on the following definition of {\tt Pair}:
\begin{verbatim}
   class Pair<X,Y> {
       X left;
       Y right;
       Pair(X left, Y right) {
           this.left = left;
           this.right = right;
       }
   }
\end{verbatim}

\end{problem}
\newpage
\ 
\newpage
\ 
\end{document}

