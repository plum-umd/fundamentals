% You must \input this file into a file the has previously
% defined the \ifrubric conditional.

\documentclass[12pt]{article}                   % -*- latex -*-

%%%%%%%%%%%%%%%%%%%%%%%%%%%%%%%%%%%%%%%%%%%%%%%%%%%%%%%%%%%%%%%%%%%%%%%%%%%%%%%
\usepackage{./exam}
\usepackage{type1cm} % We are using scalable fonts. Really.
\usepackage{alltt,times,comment,amsmath}
\usepackage{alltt}
\usepackage{ct} % tight CM font for \tt
\usepackage[usenames,dvipsnames]{xcolor}
\newexercise{exercise}{Exercise}[subsection]

\makeatletter
\newcommand{\ie}{\mbox{\emph{i.e.}}}    % \mbox keeps the last period from
\newcommand{\Ie}{\mbox{\emph{I.e.}}}    % looking like an end-of-sentence.
\newcommand{\etc}{\emph{etc.}}

\newcommand{\itum}[1]{\item{\bf #1}\\*}

\newcommand{\var}[1]{\textrm{\textit{#1}}}

\newenvironment{inset}
 {\bgroup\parskip=1ex plus 1ex\begin{list}{}%
        {\topsep=0pt\rightmargin\leftmargin}%
        \item[]}%
 {\end{list}\leavevmode\egroup\global\@ignoretrue}

\newenvironment{insetverb}%
  {\begin{inset}\begin{verbatim}}%
  {\end{verbatim}\end{end}}%

\def\pts#1{\marginpar{\footnotesize \raggedright  \fbox{#1 {\sc Points}}}}
%\def\pts#1{\relax}

\ifrubric
\newenvironment{solution}{\color{Red}}{}
\else
\excludecomment{solution}
\fi
\newexercise{problem}{Problem}

\newcommand\code[1]{\texttt{#1}}

% Works in math mode; all special chars remain special; cheaper than \cd.
% Will not be correct size in super and subscripts, though.
\newcommand{\ex}[1]{\mbox{\ttt #1}} 
\makeatother

%%%%%%%%%%%%%%%%%%%%%%%%%%%%%%%%%%%%%%%%%%%%%%%%%%%%%%%%%%%%%%%%%%%%%%%%%%%%%%%
\begin{document}
\vspace*{-1.5cm}
\centerline{\Large CSU2510H Exam 2 \ifrubric (SOLUTION) \fi-- Spring 2013}

\vspace{0.5cm}

\ifrubric\relax\else
\begin{center}
\begin{tabular}{l@{\qquad}l}
Name:                        & \rule{174pt}{1pt} \\[.5cm]
Student Id (last 4 digits):  & \rule{174pt}{1pt} \\[.5cm]
%Section (morning, honors or afternoon):           & \rule{174pt}{1pt} \\[.5cm]
\end{tabular}
\end{center}
\fi

\noindent\begin{minipage}{8cm}\sloppy
\begin{itemize}

\item All problems must be done in Java. You may use any Java we have
  used in class and in lab; anything else must be defined.

\item You may write {\tt {\slshape c} $\rightarrow$ {\slshape e}} as
  shorthand for writing {\tt
    Tester.checkExpect}; the {\tt Examples} class and {\tt test}
  method around the tests are not required.

\item To add a method to an existing class definition, you
  may write just the method and indicate the appropriate class name
  rather than re-write the entire class definition.

\item We expect data \emph{and} interface definitions.

\item If an interface is given to you, you do not need to repeat the
  contract and purpose statements in your implementations.  Likewise,
  you do not need to repeat any test cases given to you, but you
  should add tests wherever appropriate.

%% DVH: This info is already in the table.
% \item You may obtain a maximum of 55 points: 50 for the first six
% problems; and five extra-credit points for the final problem.

% \item The extra credit problem is \emph{all or nothing};  no partial
%   credit will be awarded.

\item Unless specifically requested, templates and super classes are
  \emph{not} required.

\item Some basic test taking advice: Before you start answering
any problems, read \emph{every} problem, so your brain can  think
about the harder problems in the background while you knock off the easy ones.
\end{itemize}
\bigskip

\emph{Good luck!}
\end{minipage}\hfil\begin{minipage}[t]{6cm}
\rule{1cm}{0pt}\begin{tabular}{|c|l|@{/}r|}
\hline
{\bf Problem} & Points & out of \\ \hline
1 & & 20\\ \hline
2 & & 19\\ \hline
Extra & & 5 \\ \hline
{\bf Total} & & 39+5 \\ \hline
%{\bf Base}  & \multicolumn{2}{l|}{{\bf 56}} \\ \hline
\end{tabular}
\end{minipage}

\vfill\thispagestyle{empty}
\newpage

\begin{problem} \pts{20}

 Binary tree problem

\ifrubric
\else
\newpage
[Here is some more space for the previous problem.]
\fi
\newpage
\newpage

\textbf{Extra Credit}

\end{problem}

\ifrubric
\else
\newpage
[Here is some more space for the previous problem.]
\fi
\newpage

%%%%%%%%%%%%%%%%%%%%%%%%%%%%%%%%%%%%%%%%%%%%%%%%%%%%%%%%%%%%%%%%%%%%%%%%%%%%%%%

\begin{problem} \pts{20}
\noindent
Tobin-Hochstadt needs to automatically manage papers to make it
easier to review them.  He wants to represent a paper in the following
way:

A \emph{Section} has a title and a list of subsections, which are just
sections themselves.  A paper is therefore just a big section.  Here
is an example of a paper (and, of course, an example of a section):

\begin{verbatim}
   Fundies 2, Day 1
   1. OO = struct + fun
   1.1. Rocket
   1.2. Moon
   2. Design
   2.1. Announce
   2.1.1. Assign
   3. End
\end{verbatim}
Note that the section numbers are \emph{NOT} part of the document!

\bigskip
%\vskip{2}

\begin{enumerate}

\item Develop data, class, and interface defintions for sections.  

\newpage

\noindent
\item Extend your defininition of \texttt{Section} to support an
\texttt{accept} method that take \texttt{SectionVisitor}s over to
compute over sections.  You will need to define the
\texttt{SectionVisitor} interface.  You may need to define additional
visitors as well, depending on your original data definition.

\newpage

\noindent
\item Write a \texttt{SectionVisitor} that produces the ``outline'' of a section, with the
section numbers.  The result should be an \texttt{ArrayList} of
\texttt{String}s, as seen in the initial example for this problem.

\ifrubric
\else
\newpage
[Here is some more space for the previous problem.]
\fi
\newpage

\noindent
\item Here is a defintion for the \texttt{equals} and \texttt{hashCode}
methods for \texttt{Section}s:

\begin{verbatim}
Boolean sameSec(Section s) { ...}

boolean equals(Object o) {
    if (o instanceof Section)
        return this.sameSec((Section)o);
    else
        return false;
}

int hashCode() { return 42; }
\end{verbatim}

\noindent
Develop the \texttt{sameSec} method, which \texttt{equals} relies on.

\ifrubric
\else
\newpage
[Here is some more space for the previous problem.]
\fi
\newpage

\noindent
\item The definition for \texttt{hashCode} that we have presented is very
simple, but it will reduce the performance of hash tables of
\texttt{Section}s.  Develop a better version of the \texttt{hashCode}
method.  Make sure that it can distinguish (some) \texttt{Section}s
with the same title but different contents, as well as (some)
\texttt{Section}s with different titles but the same contents.

\ifrubric
\else
\newpage
[Here is some more space for the previous problem.]
\fi
\newpage
\newpage
\end{enumerate}
\end{problem}


%%%%%%%%%%%%%%%%%%%%%%%%%%%%%%%%%%%%%%%%%%%%%%%%%%%%%%%%%%%%%%%%%%%%%%%%%%%%%%%

\begin{problem} \pts{20}

Computation/thunk problem

\ifrubric
\else
\newpage
[Here is some more space for the previous problem.]
\fi
\newpage
\newpage

\end{problem}

\ifrubric
\else
\newpage
[Here is some more space for the previous problem.]
\fi
\newpage


%%%%%%%%%%%%%%%%%%%%%%%%%%%%%%%%%%%%%%%%%%%%%%%%%%%%%%%%%%%%%%%%%%%%%%%%%%%%%%%
\end{document}


