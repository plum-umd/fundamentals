\documentclass[12pt]{article}
\usepackage{fullpage}
\usepackage{xcolor,censor}
\usepackage{fancyvrb}
\usepackage{fontspec}
\usepackage{alltt}

\setmainfont{Times New Roman}

\censorruleheight=0ex
%\StopCensoring

\begin{document}

\begin{center}
  {\Large {\bf CMSC 131A, Final Exam}\\
    \censor{{\bf SOLUTION}} \ \\
    \ \\
    Fall 2018 \ \\
  }
\end{center}

\vspace{2em}

\noindent
NAME:\verb|________________________________________________________|

\vspace{2em}

\noindent
UID: \verb|__________________________________________|

\vspace{2em}

\begin{center}
\begin{tabular}{| c | c |}
  \hline
  Question & Points \\ \hline \hline
  1 & 10 \\ \hline
  2 & 10 \\ \hline
  3 & 15 \\ \hline
  4 & 15 \\ \hline
  5 & 15 \\ \hline
  6 & 15 \\ \hline
  7 & 15 \\ \hline\hline
  Total: & 95\\
  \hline
\end{tabular}
\end{center}

\noindent
This test is open-book, open-notes, but you may not use any computing
device other than your brain and may not communicate with anyone.
You have 120 minutes to complete the test.

\vskip 1em

\noindent
The phrase ``design a program'' or ``design a function'' means follow
the steps of the design recipe.  Unless specifically asked for, you do
not need to provide intermediate products like templates or stubs,
though they may be useful to help you construct correct solutions.

\vskip 1em

\noindent
You may use any of the data definitions given to you within this exam
and do not need to repeat their definitions.

\vskip 1em

\noindent
Unless specifically instructed otherwise, you may use any built-in ISL+
functions or data types.

\vskip 1em

\noindent
When writing tests, you may use a shorthand for writing check-expects
by drawing an arrow between two expressions to mean you expect the
first to evaluate to same result as the second.  For example, you may
write \verb|(add1 3)| $\rightarrow$ \verb|4| instead of
\verb|(check-expect (add1 3) 4)|.

\newpage

\noindent
{\bf Problem 1 (10 points).}
%
For the following program, write out each step of computation. At each
step, underline the expression being simplified. Label each step as
being ``arithmetic'' (meaning any built-in operation),
``conditional'', ``plug'' (for plugging in an argument for a function
parameter), or ``constant'' for replacing a constant with its value.


\begin{verbatim}
(define S 1)
(define (w p) (< S p))

(cond [(w 9) (* 8 2)]
      [else 9])
\end{verbatim}


\begin{SaveVerbatim}{VerbEnv}

SOLUTION:

(cond [(w 9) (* 8 2)] [else 9])      --[plug]-->
       ^^^^^
(cond [(< S 9) (* 8 2)] [else 9])    --[const]-->
          ^
(cond [(< 1 9) (* 8 2)] [else 9])    --[arith]-->
       ^^^^^^^
(cond [#true (* 8 2)] [else 9])      --[cond]-->
^^^^^^^^^^^^^^^^^^^^^^^^^^^^^^^
(* 8 2)                              --[arith]-->
^^^^^^^
16
\end{SaveVerbatim}

\censor{%
\noindent\BUseVerbatim{VerbEnv}}

\newpage

\noindent
{\bf Problem 2 (10 points).}
%
For the following structure definition, list the names of every
function it creates.  For each function, classify it as being either a
constructor, accessor, or predicate.

\begin{verbatim}
(define-struct roger (waters rabbit))
\end{verbatim}

\begin{SaveVerbatim}{VerbEnv}  
SOLUTION:

- make-roger   : Constructor
- roger-waters : Accessor
- roger-rabbit : Accessor
- roger?       : Predicate
\end{SaveVerbatim}

\censor{%
\noindent\BUseVerbatim{VerbEnv}}


\newpage

\noindent
{\bf Problem 3 (15 points).}
%
Imagine designing the following two functions.  Now design an
abstraction of both and define each original function in terms of that
abstraction.  (It's not necessary to give the ``unabstracted''
definitions unless it's helpful to you.)
\begin{verbatim}
;; find-even : [Listof Number] -> [Maybe Number]
;; Find first even number, if there is one (or #false)
(check-expect (find-even '(1 2 3 4)) 2)
(check-expect (find-even '(1 3 5)) #false)

;; find-long : [Listof String] -> [Maybe String]
;; Find first string longer than 5 letters, if there is one (or #false)
(check-expect (find-long '("a" "really" "long")) "really")
(check-expect (find-long '("a" "b" "c")) #false)
\end{verbatim}
Recall that {\tt [Maybe X]} is defined as:
\begin{verbatim}
;; A [Maybe X] is one of:
;; - X
;; - #false
\end{verbatim}

\begin{SaveVerbatim}{VerbEnv}


SOLUTION:

;; find : [Listof X] [X -> Boolean] -> [Maybe X]
;; Find first element satisfying the predicate, if there is one (or #false)
(define (find lon pred)
  (cond [(empty? lon) #false]
        [(cons? lon)
         (cond [(pred (first lon)) (first lon)]
               [else (find-even (rest lon))])]))


(define (find-even lon)
  (find lon even?))

(define (find-long los)
  (find los (lambda (s) (> (string-length s) 5))))
\end{SaveVerbatim}

%\censor{\vspace{-2.5in}}
\censor{%
\noindent\BUseVerbatim{VerbEnv}}


\newpage

[Space for problem 3.]


\newpage

\noindent
{\bf Problem 4 (15 points).}
%
Here is a data definition for binary trees:
\begin{verbatim}
;; A [BT X] is one of:
;; - (make-leaf)
;; - (make-node X [BT X] [BT X])
(define-struct node (val left right))
(define-struct leaf ())
\end{verbatim}
Design a function that consumes a binary tree of strings and
constructs a single string that results from concating the strings
in left to val to right order.  For example:
\begin{verbatim}
(define l (make-leaf))
(define t (make-node "b" (make-node "a" l l) (make-node "c" l l)))
\end{verbatim}
The function should produce {\tt "abc"} on {\tt t}.

\begin{SaveVerbatim}{VerbEnv}


SOLUTION:

\end{SaveVerbatim}

\censor{%
\noindent
\BUseVerbatim{VerbEnv}}


%% Using an accumulator-based design, design the program {\tt to10} that
%% converts a list of digits into its correspdonding number.  The
%% elements are listed in order of most to least significance,
%% so for example, {\tt (to10 '(9 1 2 3))} should produce 9123.
%% \begin{verbatim}
%% ;; A Digit is a Natural in [0,9]

%% ;; to10 : [Listof Digit] -> Number
%% ;; Convert digits (in order of most to least significance) to a number
%% (check-expect (to10 '(1 2 3)) 123)
%% (check-expect (to10 '(7 2 8)) 728)
%% \end{verbatim}
%% Here is a hint, 728 can be computed as {\tt (+ 8 (* 10 (+ 2 (* 10 (+ 7 0)))))}.
%% A design that does not use an accumulator will receive only partial credit.

%% \begin{SaveVerbatim}{VerbEnv}


%% SOLUTION:

%% (define (to10 lod)
%%   (local [(define (to10/a lod a)
%%             (cond [(empty? lod) a]
%%                   [(cons? lod)
%%                    (to10/a (rest lod)
%%                            (+ (first lod)
%%                               (* 10 a)))]))]                              
%%     (to10/a lod 0)))
%% \end{SaveVerbatim}

%% \censor{%
%% \noindent
%% \BUseVerbatim{VerbEnv}}



\newpage 
\noindent
{\bf Problem 5 (15 points).}
%
Design the following variant of {\tt map} that takes a binary function
and \emph{two} lists that are assumed to be of equal length and
produces list of results formed by applying the function to the
elements of each list.  Only tests are given, you have to come up with
the rest:
\begin{verbatim}
(check-expect (2map < '(1 4 3) '(3 1 2)) '(#true #false #false))
(check-expect (2map string-ith '("fred" "wilma") '(1 0)) '("r" "w"))
\end{verbatim}


\begin{SaveVerbatim}{VerbEnv}


SOLUTION:

;; 2map : [X Y -> Z] [Listof X] [Listof Y] -> [Listof Z]
;; Binary variant of map
;; ASSUME: lists are equal length
(define (2map f xs ys)
  (cond [(empty? xs) '()]
        [(cons? xs)
         (cons (f (first xs) (first ys))
               (2map f (rest xs) (rest ys)))]))
\end{SaveVerbatim}

\censor{%
\noindent
\BUseVerbatim{VerbEnv}}



\newpage
\noindent
{\bf Problem 6 (15 points).}  
%
During your internship at the University of Maryland's Center for
Bioinformatics and Computational Biology (CBCB), you're asked to
improve code written by last year's intern.  That intern designed a
program for computing the \emph{reverse complement} of DNA strands.
%
A DNA strand is an arbitrarily long sequence of bases, which are
either: C, G, A, or T.  A and T are complements of each other, as are
C and G.  The reverse complement of a DNA strand is formed by
reversing the strand and taking the complement of each symbol.  For
example, the reverse complement of AAAACCCGGT is ACCGGGTTTT.
%
Last year's intern came up with the following:
\begin{verbatim}
;; A DNA is a [Listof Base]
;; A Base is one of: "A", "C", "G", "T"
;; Interp: DNA is a sequence of nucleotide bases

;; rev-comp : DNA -> DNA
;; Compute the reverse complement of the given strand
(check-expect (rev-comp (list "A" "A" "C" "G" "T"))
              (list "A" "C" "G" "T" "T"))
(define (rev-comp dna)
  (reverse (map comp dna)))

;; comp : Base -> Base
;; Compute the complement of the given base
(check-expect (comp "A") "T")
(check-expect (comp "T") "A")
(check-expect (comp "C") "G")
(check-expect (comp "G") "C")
(define (comp b)
  (cond [(string=? b "A") "T"]
        [(string=? b "T") "A"]
        [(string=? b "C") "G"]
        [(string=? b "G") "C"]))
\end{verbatim}
You notice that {\tt rev-comp} traverses the DNA sequence twice, once
with {\tt map} and once with {\tt reverse}.  Improve this design by
revising {\tt rev-comp} to use an accumulator to traverse the DNA
sequence just once.

\newpage

\noindent
[Space for problem 6.]

\begin{SaveVerbatim}{VerbEnv}


SOLUTION:

(define (rev-comp dna)
  (local [;; DNA DNA -> DNA
          ;; ACCUM: the reverse complement of the sequence so far
          (define (rev-comp/a dna rcomp)
            (cond [(empty? dna) rcomp]
                  [(cons? dna)
                   (rev-comp/a (rest dna)
                               (cons (comp (first dna)) rcomp))]))]
     (rev-comp/a dna '())))                      
\end{SaveVerbatim}

\censor{%
\noindent
\BUseVerbatim{VerbEnv}}



\newpage


\noindent
{\bf Problem 7 (15 points).} Design a program that takes two
strings and counts how many times the first string occurs within the
second, {\bf including possibly overlapping occurences}.
For example,
\begin{itemize}
\item {\tt "ho"} occurs three times in {\tt "ho ho ho"}, while
\item {\tt "ho ho"} occurs {\bf twice} in {\tt "ho ho ho"} (notice that occurences
  overlap on the middle {\tt "ho"}).
\end{itemize}
You may assume the first string is non-empty.
%
It may be helpful to use the {\tt substring} function.  Recall: {\tt
  (substring "hello world" 1 5)} produces {\tt "ello"} and {\tt
  (substring "hello world" 4)} produces {\tt "o world"}.


\begin{SaveVerbatim}{VerbEnv}


SOLUTION:

;; occurs : String String -> Natural
;; Count (possibly overlapping) occurences of first string in second
;; Assume: first string is non-empty
;; Termination: each recursive call is on a strictly
;; smaller s2, therefore eventually reaching first cond clause (trivial case).
(check-expect (occurs "ho" "ho ho ho") 3)
(check-expect (occurs "ho ho" "ho ho ho") 2)
(define (occurs s1 s2)
  (local [(define s1-len (string-length s1))]
    (cond [(< (string-length s2) s1-len) 0]
          [(string=? s1 (substring s2 0 s1-len))
           (add1 (occurs s1 (substring s2 1)))]
          [else
           (occurs s1 (substring s2 1))])))
\end{SaveVerbatim}

\censor{%
\noindent
\BUseVerbatim{VerbEnv}}



%% \newpage

%% \noindent
%% {\bf Problem 8 (15 points).} An association list is a way of
%% representing associations between pairs of elements.  For example, a
%% dictionary is an association between words (strings) and their
%% definition (also strings).  The final grades for this class might be
%% represented as an association between student IDs (numbers) and letter
%% grades (strings).  Here is a parameterized data definition for
%% associations:
%% \begin{verbatim}
%% ;; An [Assoc X Y] is one of:
%% ;; - '()
%% ;; - (cons (list X Y) [Assoc X Y])
%% ;; Interp: a collection of associations where each element is a
%% ;; two-element list of associated elements.
%% \end{verbatim}
%% Here are two functions for looking up values in different kinds of associations:
%% \begin{verbatim}
%% ;; lookup-dict : [Assoc String String] String -> [Maybe String]
%% ;; Lookup meaning of a word in a dictionary
%% (check-expect (lookup-dict (list (list "dog" "goodboy")) "dog") "goodboy")
%% (check-expect (lookup-dict (list (list "dog" "goodboy")) "cat") #false)
%% (define (lookup-dict d w)
%%   (cond [(empty? d) #false]
%%         [(string=? (first (first d)) w) (second (first d))]
%%         [else (lookup-dict (rest d) w)]))

%% ;; lookup-grade : [Assoc Number String] Number -> [Maybe String]
%% ;; Lookup letter grade for given student ID
%% (check-expect (lookup-grade (list (list 1112 "A++")) 1112) "A++")
%% (check-expect (lookup-grade (list (list 1112 "A++")) 9999) #false)
%% (define (lookup-grade d w)
%%   (cond [(empty? d) #false]
%%         [(= (first (first d)) w) (second (first d))]
%%         [else (lookup-grade (rest d) w)]))
%% \end{verbatim}
%% Design an abstraction of these two functions and redefine these functions in terms of this abstraction.
%% (You do not need to write additional tests; your solution should not use {\tt equal?}.)

%% \newpage
%% [Space for problem 8.]

\end{document}

