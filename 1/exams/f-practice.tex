\documentclass[12pt]{article}
\usepackage{fullpage}
\usepackage{xcolor,censor}
\usepackage{fancyvrb}
\usepackage{fontspec}
\usepackage{alltt}

\setmainfont{Times New Roman}

\censorruleheight=0ex
%\StopCensoring

\begin{document}

\begin{center}
  {\Large {\bf CMSC 131A, Final Exam (Practice)}\\
    \censor{{\bf SOLUTION}} \ \\
    \ \\
    Fall 2017 \ \\
  }
\end{center}

\vspace{2em}

\noindent
NAME:\verb|________________________________________________________|

\vspace{2em}

\noindent
UID: \verb|__________________________________________|

\vspace{2em}

\begin{center}
\begin{tabular}{| c | c |}
  \hline
  Question & Points \\ \hline \hline
  1 & 10 \\ \hline
  2 & 10 \\ \hline
  3 & 15 \\ \hline
  4 & 12 \\ \hline
  5 & 15 \\ \hline
  6 & 15 \\ \hline
  7 & 15 \\ \hline
  8 & 20 \\ \hline\hline
  Total: & 112\\
  \hline
\end{tabular}
\end{center}

\noindent
This test is open-book, open-notes, but you may not use any computing
device other than your brain and may not communicate with anyone.
You have 50 minutes to complete the test.

\vskip 1em

\noindent
The phrase ``design a program'' or ``design a function'' means follow
the steps of the design recipe.  Unless specifically asked for, you do
not need to provide intermediate products like templates or stubs,
though they may be useful to help you construct correct solutions.

\vskip 1em

\noindent
You may use any of the data definitions given to you within this exam
and do not need to repeat their definitions.

\vskip 1em

\noindent
Unless specifically instructed otherwise, you may use any built-in ISL+
functions or data types.

\vskip 1em

\noindent
When writing tests, you may use a shorthand for writing check-expects
by drawing an arrow between two expressions to mean you expect the
first to evaluate to same result as the second.  For example, you may
write \verb|(add1 3)| $\rightarrow$ \verb|4| instead of
\verb|(check-expect (add1 3) 4)|.

\newpage

\noindent
{\bf Problem 1 (10 points).}
%
For the following program, write out each step of computation. At each
step, underline the expression being simplified. Label each step as
being ``arithmetic'' (meaning any built-in operation),
``conditional'', ``plug'' (for plugging in an argument for a function
parameter), or ``constant'' for replacing a constant with its value.


\begin{verbatim}
(define T 7)
(define (q z) (sqr z))
(cond [(> T 3) (q 4)]
      [else 9])
\end{verbatim}

\newpage

\noindent
{\bf Problem 2 (10 points).}
%
For the following structure definition, list the names of every
function it creates.  For each function, classify it as being either a
constructor, accessor, or predicate.

\begin{verbatim}
(define-struct zip (pad msg))
\end{verbatim}

\begin{SaveVerbatim}{VerbEnv}  
SOLUTION:

- make-zip  : Constructor
- zip-pad  : Accessor
- zip-msg   : Accessor
- zip?      : Predicate
\end{SaveVerbatim}


\newpage

\noindent
{\bf Problem 3 (15 points).}
%
Here are some data definition relevant to representing a dictionary,
which associates words with the their definitions:

\begin{verbatim}
;; A Dict is one of:
;; - '()
;; - (cons (list String String) Dict)
;; Interp: a collection of definitions where each element is a 
;; two-element list of a word (first) and its meaning (second).
\end{verbatim}

\noindent
Design the following function:

\begin{verbatim}
;; update : String String Dict -> Dict
;; Update entry for given word and meaning, if it exists
;; Add the entry if it does not
\end{verbatim}

\begin{SaveVerbatim}{VerbEnv}

SOLUTION:

(check-expect (update "a" "b" '()) (list (list "a" "b")))
(check-expect (update "a" "c" (list (list "a" "b")))
              (list (list "a" "c")))
(check-expect (update "a" "e" (list (list "c" "b") (list "a" "d"))) 
              (list (list "c" "b") (list "a" "e")))
(define (update w m d)
  (cond [(empty? d) (list (list w m))]
        [(cons? d)
         (if (string=? (first (first d)) w)
             (cons (list w m) d)
             (cons (first d) (lookup w (rest d))))]))

\end{SaveVerbatim}

%\censor{\vspace{-2.5in}}
\censor{%
\noindent\BUseVerbatim{VerbEnv}}

\newpage

\noindent
{\bf Problem 4 (12 points).}
%
For each of the following functions, provide the most general {\bf type} signature
in Typed Racket that correctly describes the function:

\begin{alltt}
(: method : \censor{Number -> Number                                    })

(define (method x)
  (+ (sqr x) 2))

(: raekwon : \censor{[Listof Number] -> Boolean                         })

(define (raekwon x)
  (ormap positive? x))

(: rza : \censor{[Listof String] [Listof Number] -> [Listof Number]     })

(define (rza x y)
  (cond [(empty? x) y]
        [(cons? x)
         (cons (string-length (first x))
               (rza (rest x) y))]))

(: gza : \censor{(String Number -> Number) -> Number                    })

(define (gza x)
  (foldr x 0 (list "a" "b" "c")))

(: odb : \censor{(Number -> Number) -> (Number -> Number)               })

(define (odb x)
  (lambda (y)   
    (/ (- (x (+ y 0.001))
          (x (- y 0.001)))
       (* 2 0.001))))
    

(: ghostface : \censor{[X Y] [Listof X] [X -> Y] -> [Listof Y]          })

(define (ghostface x y)
  (cond [(empty? x) '()]
        [(cons? x)
         (cons (y (first x)) 
               (ghostface (rest x) y))]))
\end{alltt}


\newpage

\noindent
{\bf Problem 5 (15 points).}
%
Here is the design of a program {\tt w-avg} that computes the weighted
average of a list of numbers and a list of weights.

Design an accumulator variant of {\tt w-avg} that maintains two
accumulators: the weighted sum of scores and the sum of weights.
Redefine {\tt w-avg} in terms of your accumulator design.

For example, let's say a class grade is based on two midterms and a
project where the project is worth twice as much as the midterms.  A
student who gets an 80 and a 70 on the midterms and a 90 on the
project would have a weighted average of 82.5: $((1 \times 80) + (1
\times 70) + (2 \times 90)) / 4$, which can be computed with {\tt
  (w-avg (list 80 70 90) (list 1 1 2))}.

In the accumulator version, after all elements of the list have been
seen, the first accumulator would be $330 = 80+70+2\times 90$) and
the second accumulator would be $4 =1+1+2$.

\begin{verbatim}
;; w-avg : [Listof Number] [Listof Number] -> Number
;; Compute the weighted average of a list of numbers and weights
;; Assume: lists have the same length (and non-empty)
(check-expect (w-avg (list 70 80 90) (list 1 1 2)) 82.5)
(define (w-avg lon ws)
  (/ (w-sum lon ws) 
     (foldr + 0 ws)))

;; w-sum : [Listof Number] [Listof Number] -> Number
;; Sum the list of numbers according to given weights
;; Assume: lists have same length
(check-expect (w-sum (list 70 80 90) (list 1 1 2)) (+ 70 80 180))
(define (w-sum lon ws)
  (cond [(empty? lon) 0]
        [(cons? lon)
         (+ (* (first lon) (first ws))
            (w-sum (rest lon) (rest ws)))]))
\end{verbatim}

\noindent
Your solution should not need a helper function like {\tt w-sum}.

\begin{SaveVerbatim}{VerbEnv}


SOLUTION:


;; Alt: define w-total function for denominator

;; w-total : [Listof Number] -> Number
;; Total a list of weights
(define (w-total ws)
  (cond [(empty? ws) 0]
        [(cons? ws)
         (+ (first ws) 
            (w-total (rest ws)))]))
\end{SaveVerbatim}

\censor{%
\noindent
\BUseVerbatim{VerbEnv}}


\newpage

[Space for problem 5.]


\newpage 
\noindent
{\bf Problem 6 (15 points).}
%
Here is a parametric data definition for a tree of elements:
\begin{verbatim}
;; A [Tree X] is one of:
;; - (make-leaf)
;; - (make-node X [Tree X] [Tree X])
;; Interp: a binary tree that is either empty (a leaf), or non-empty (a node)
;; with an element and two sub-trees.
(define-struct leaf ())
(define-struct node (elem left right))
\end{verbatim}

\noindent
Here is a function that counts the elements in a tree:
\begin{verbatim}
;; tree-count : [X] . [Tree X] -> Natural
;; Count all the elements in a tree
(check-expect (tree-count (make-leaf)) 0)
(check-expect (tree-count (make-node 7 
                                     (make-node 9 (make-leaf) (make-leaf))
                                     (make-node 2 (make-leaf) (make-leaf))))
              3)
(define (tree-count bt)
  (cond [(leaf? bt) 0]
        [(node? bt)
         (add1 (+ (tree-count (node-left bt))
                  (tree-count (node-right bt))))]))
\end{verbatim}
Here is an abstraction function for trees that is similar to
{\tt foldr} for lists, but works on trees:
\begin{verbatim}
;; tree-fold : [X Y] . [X Y Y -> Y] Y [Tree X] -> Y
;; The fundamental abstraction function for trees
(define (tree-fold f b bt)
  (cond [(leaf? bt) b]
        [(node? bt)
         (f (node-elem bt)
            (tree-fold f b (node-left bt))
            (tree-fold f b (node-right bt)))]))
\end{verbatim}
Give an equivalent definition of {\tt tree-count} in terms of {\tt
  tree-fold}.  (Just provide the code.)

\vspace{1em}
\noindent
[Provide your answer on the next page.]

\begin{SaveVerbatim}{VerbEnv}


SOLUTION:

(define (tree-elems bt)
  (tree-fold (lambda (x l r) (cons x (append l r))) '() bt))

\end{SaveVerbatim}

\censor{%
\noindent
\BUseVerbatim{VerbEnv}}

\newpage

\noindent
[Space for problem 6.]


\newpage
\noindent
{\bf Problem 7 (15 points).}  
%
During your internship at the University of Maryland's Center for
Bioinformatics and Computational Biology (CBCB), you're asked to
design a program for transcribing DNA strands into RNA strands.
%
A DNA strand is an arbitrarily long sequence of bases, which are
either: C, G, A, or T; an RNA strand is an arbitrarily long sequences
of either: C, G, A, or U.  The \emph{transcription} of DNA to RNA
replaces each occurrence of T with U.

Design a program that, given a DNA strand, computes its transcription
to RNA.

\newpage


\noindent
{\bf Problem 8 (20 points).} Here is the data definition for graphs we studied in class:
\begin{verbatim}
;; A Graph is [Listof (cons String [Listof String])]
;; Interp: a graph is a list of nodes and neighbors reachable from the node

;; A Path is a [Listof String]
;; Interp: series of nodes neighboring each other that connect a source
;; to destination in some graph.
\end{verbatim}

\noindent
Design the following function:

\begin{verbatim}
;; all-paths : Graph String String -> [Listof Path]
;; Produce all non-cyclic paths from src to dst in g.
;; Assume: both nodes exist in graph.
(define (all-paths g src dst) '())
\end{verbatim}

\noindent
You may assume the following function is already defined and works correctly:

\begin{verbatim}
;; neighbors : Graph String -> [Listof String]
;; Produce the list of immediate neighbors of given node name.
;; Assume: node exists in graph.
\end{verbatim}

\noindent
Note: this function is tricky to test since the tests shouldn't
specify the order of the paths in the list.  You may ignore this
subtly and just write tests as though a {\tt check-expect} will pass
if the checked and expected expressions produce lists with the same
elements in any order, e.g. {\tt (check-expect (list 1 2) (list 2 1))}
would pass.

\newpage
[Space for problem 8.]
\end{document}

